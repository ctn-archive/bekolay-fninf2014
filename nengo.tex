\documentclass{frontiersSCNS}

%%%%% Some customizations

\usepackage{listings}
\usepackage{inconsolata}
\graphicspath{{figures/}}

%%%%% Frontiers boilerplate

\usepackage{url,lineno}
\linenumbers

\copyrightyear{2013}
\pubyear{2013}

\def\journal{Neuroinformatics}
\def\DOI{}
\def\articleType{Methods Article}
\def\keyFont{\fontsize{8}{11}\helveticabold }
\def\firstAuthorLast{Bekolay {et~al.}}
\def\Authors{Trevor Bekolay\,$^{1,*}$, James Bergstra\,$^{1}$,
  Eric Hunsberger\,$^{1}$, Travis DeWolf\,$^{1}$, Terrence C. Stewart\,$^{1}$,
  Daniel Rasmussen\,$^{1}$, Xuan Choo\,$^{1}$, Aaron Voelker\,$^{1}$,
  and Chris Eliasmith\,$^{1}$}
\def\Address{$^{1}$Center for Theoretical Neuroscience,
  University of Waterloo, Waterloo, ON, Canada}
\def\corrAuthor{Trevor Bekolay}
\def\corrAddress{David R. Cheriton School of Computer Science,
  University of Waterloo, 200 University Avenue West,
  Waterloo, ON, N2L 3G1, Canada}
\def\corrEmail{tbekolay@uwaterloo.ca}

%%%%% Document begins

\begin{document}
\onecolumn
\firstpage{1}

\title[Nengo in Python]{Nengo: A Python tool for
  building large-scale functional brain models}
\author[\firstAuthorLast ]{\Authors}
\address{}
\correspondance{}
\extraAuth{}
\topic{Python in Neuroscience II}

\maketitle

\begin{abstract}
  Neuroscience currently lacks
  a comprehensive theory of
  how cognitive processes can be
  implemented in a biological substrate.
  The Neural Engineering Framework (NEF)
  proposes one such theory,
  but has not yet gathered
  significant empirical support,
  partly due to the technical
  challenge of building and simulating
  large-scale models with the NEF.
  Nengo is a software tool
  that can be used to build and simulate
  large-scale models based on the NEF;
  currently, it is the primary resource
  for both teaching how the NEF
  is used,
  and for doing research
  that generates specific
  NEF models
  to explain experimental data.
  A previous version of Nengo
  implemented in Java was used
  to create Spaun,
  the world's largest functional
  brain model \citep{eliasmith2012}.
  Simulating Spaun
  highlighted limitations
  in Nengo's ability to
  support model construction with simple syntax,
  to simulate large models quickly,
  and to collect large amounts of data
  for subsequent analysis.
  This paper describes a new version of Nengo
  implemented in Python that overcomes these limitations.
  It uses simple and extendable syntax,
  simulates a benchmark model on the scale of Spaun
  50 times faster than the previous implementation,
  and has a flexible mechanism
  for collecting simulation results.

  \tiny \keyFont{\section{Keywords:} Neural Engineering
    Framework, Nengo, Python, neuroscience, theoretical neuroscience,
    control theory, simulation}
\end{abstract}

\section{Introduction}

Modeling the human brain
is one of the greatest
scientific challenges of our time.
Computational neuroscience
has made significant advancements
from simulating low-level biological parts in great detail,
to solving high-level problems that humans find difficult;
however, we still lack a mathematical account of
how biological components
implement cognitive functions such as sensory processing,
memory formation, reasoning, and motor control.
Much work has been put into neural simulators
that attempt to recreate neuroscientific
data in precise detail with the thought that
cognition will emerge by connecting
detailed neuron models according
to the statistics of biological synapses \citep{markram2006}.
However, cognition has not yet emerged
from data-driven large scale models,
and there are good reasons to think
that cognition may never emerge \citep{trujillo-inpress}.
At the other end of the spectrum,
cognitive architectures \citep{anderson2004}
and machine learning approaches \citep{hinton2006}
have solved high-dimensional statistical problems,
but do so without respecting biological constraints.

Nengo is a neural simulator
based on a theoretical framework proposed
by \citet{eliasmith2003}
called the Neural Engineering Framework
(NEF).
The NEF is a large-scale modeling approach
that can leverage single neuron models
to build neural networks with
demonstrable cognitive abilities.
Nengo and the NEF has been used to build
increasingly sophisticated neural subsystems
for the last decade
(e.g., path integration [\citealp{conklin2005}],
working memory [\citealp{singh2006}],
list memory [\citealp{choo2010}],
inductive reasoning [\citealp{rasmussen2011}],
motor control [\citealp{dewolf2011}],
decision making [\citealp{stewart2012}])
culminating recently with Spaun,
currently the world's
largest functional brain model \citep{eliasmith2012}.
Spaun is a network of 2.5 million spiking neurons
that can perform eight cognitive tasks
including memorizing lists and inductive reasoning.
It can perform any of these eight tasks
at any time by being presented
the appropriate series of images
representing the task to be performed;
for example, sequentially presenting images
containing the characters $A3[1234]$ instructs Spaun
to memorize the list $1234$.
If asked to recall the memorized list,
Spaun generates motor commands for a simulated arm,
writing out the digits $1234$.
While the tasks that Spaun performs are diverse,
all of the tasks use a common set of
functional cortical and subcortical components.
Each functional component corresponds
to a brain area that has been hypothesized
to perform those functions
in the neuroscientific literature.

The NEF provides principles
to guide the construction of
a neural model
that incorporates anatomical constraints,
functional objectives, and
dynamical systems or control theory.
Constructing models from this starting point,
rather than from single cell electrophysiology
and connectivity statistics alone,
produces simulated data that explains
and predicts a wide variety of experimental results.
Single cell activity \citep{stewart2012},
response timing \citep{stewart2009a},
and behavioral errors \citep{choo2010}
of NEF-designed models match
psychological and physiological findings
without being built specifically into the design.
These results are a consequence of
the need to satisfy functional objectives
within anatomical and neurobiological constraints.

It has often been said that neuroscience
is data-rich and theory-poor \citep{churchland1992}.
We believe that the modeling success of
large functional networks such as Spaun
suggests that the NEF
may be able to fill the theoretical void.
However, in simulating this large model,
the Java-powered Nengo simulator
was limited in terms of memory consumption,
simulation speed, and extensibility.
In order to continue to evaluate the NEF
as a theory of neural computation
by building larger and more complicated models,
we have begun the next generation
of Nengo development with a focus on speed,
extensibility, and simplicity.

This paper introduces a new version
of Nengo (PyNengo) written in Python.
The remainder of this introduction outlines
the design issues in the previous version (JavaNengo)
that motivated the reimplementation.
PyNengo is not currently meant
to completely replace JavaNengo;
JavaNengo remains a capable visual NEF
simulator and a valuable teaching tool
due to its graphical interface.
PyNengo's focus is on speed
and a simplified internal design
to support ongoing research in
large adaptive functional models.
The remainder of the paper outlines PyNengo's design,
scripting interface, and performance relative to
JavaNengo and alternatives such as Brian and NEST.

\subsection{JavaNengo limitations}

\begin{table}[!t]
\processtable{Tools bearing the name Nengo.\label{tab:nengo-vers}}
{\begin{tabular}{p{2.6cm} p{2.7cm} l p{7cm}} \toprule
\textbf{Name} & \textbf{Implementation language}
  & \textbf{Code repository} & \textbf{Purpose} \\\midrule
JavaNengo simulator & Java & \texttt{nengo\_java}
  & Simulates many types of spiking neuron models.
  Includes methods to construct and simulate models using the NEF. \\
JavaNengo GUI & Java & \texttt{nengo\_java\_gui}
  & Graphical interface to JavaNengo simulator.
  Allows modellers to drag-and-drop components built with the NEF. \\
JavaNengo scripting interface & Jython & \texttt{nengo\_jython}
  & Scripting interface to JavaNengo simulator.
  Provides shortcuts to classes and functions in the simulator.
  Also accessible through and interacts with the JavaNengo GUI. \\
PyNengo & Python & \texttt{nengo}
  & Constructs and simulates models using the NEF. \\
PyNengo OpenCL simulator & Python+PyOpenCL
  & \texttt{nengo\_ocl}
  & Simulates models constructed with PyNengo. \\\botrule
\end{tabular}}{Code repositories avilable at \url{http://github.com/ctn-waterloo/}.}
\end{table}

JavaNengo is a suite of tools composed of
a neural simulator and graphical user interface (GUI)
implemented in Java,
and a scripting interface implemented in Jython
(as detailed in Table~\ref{tab:nengo-vers}).
JavaNengo's simulator was created
with the intention of being a general
neural simulator that included
the methods of the NEF
as one of several
methods for creating neural models.
As a result, its implementation
of important NEF components
are nested in deep class hierarchies,
adding unnecessary layers of complexity
for developers wishing
to extend those NEF components.
For example,
the most often used class for making
NEF models, \texttt{NEFEnsemble},
extends from five ancestor classes
and implements seven interfaces.
Adding support for synaptic plasticity
in \texttt{NEFEnsemble}
involved adding or significantly modifying
25 Java classes \citep{bekolay2010}.

The JavaNengo GUI provides
a drag-and-drop interface in which
NEF models can be created.
However, as it was developed
before the scripting interface,
the GUI manipulates simulator objects directly,
rather than constructing a script
that manipulates those simulator objects.
Typically, modellers start by following
tutorials that use the GUI interface,
and transition to the scripting interface
in order to implement more complex models;
since the GUI modifies simulator objects directly,
modelling concepts learned in the GUI
have to be relearned in the scripting interface.

The JavaNengo scripting interface,
described in \citet{stewart2009},
provides a simpler interface
to the NEF-specific portions of the simulator.
This greatly enhanced productivity,
allowing models to be described
by short scripts.
Modellers quickly started using
the scripting interface
as their primary tool,
but still needed to access the GUI
for interactive model inspection,
and still needed to modify the simulator
to add new functionality
(e.g., to add new neuron models).
This is because Jython makes it easy
to inspect and create Java objects,
but does not make it easy to
write Java code to
inspect and create Jython objects.
Modellers, therefore, had to be
proficient in Java whether
they used the scripting interface or not.
As there are few neural modellers
that are proficient with both Java and Python syntax,
the scripting interface still lacks
convenient ways to record the results of simulations,
and to define complex experimental setups.
Additionally, Jython is always several versions
behind CPython, making it impossible
to use new language features
like dictionary comprehensions.
Jython also cannot access libraries
that contain any C code,
including all scientific libraries
that use NumPy.

Finally, perhaps the most significant limitation
of JavaNengo is its performance.
While Java has good support for threads,
it lacks native high-performance numerical libraries
for doing mathematical operations on large matrices.
Additionally, it is difficult to interface with
hardware such as graphical processing units (GPUs),
which are well-suited to the computations
required in neural simulations.

Addressing all of these issues
in the Java codebase would require
enough extensive changes
that we opted instead
to write a new version of Nengo in Python.
JavaNengo's Jython scripting interface
already used Python syntax,
and since Python is widely used
for neural simulation and data analysis,
we targeted the CPython interpreter
as the platform for the new version of Nengo.

\subsection{PyNengo}

PyNengo is a Python package for
defining and simulating
neural models using the NEF.
Its interfaces are based on NumPy,
so it integrates seamlessly
with other scientific Python tools
for interactive exploration \citep{perez2007},
scientific computation \citep{oliphant2007},
and plotting \citep{hunter2007}.
Arbitrary CPython programs
can use Nengo models,
opening up possibilities
for using neurally implemented algorithms
in web services, games, and other applications.

PyNengo has a simple object model,
which makes it easy to
document, test, and modify.
These objects are accessible
through a scripting interface
similar to JavaNengo's \citep{stewart2009},
with improvements
for easier incorporation of user-defined networks,
neuron models and learning rules.
Critically, and unlike JavaNengo,
PyNengo's scripting interface
contains functions to record simulated data
and access that data after simulation.

Model creation and simulation are decoupled,
allowing for models to be exported
to other simulation packages
(e.g., those accessible by PyNN [\citealp{davison2008}])
or for faster simulators to be used
as drop-in replacements for PyNengo's
platform-independent reference simulator.
To date, we have implemented one other simulator
that uses PyOpenCL to take advantage
of a GPU or multicore CPU.

PyNengo does not yet
replace all of JavaNengo's functionality.
While it can simulate most
models that JavaNengo can simulate,
there is currently no graphical interface for PyNengo.
Model components must be created
using Python scripts rather than
the drag-and-drop interface presented by JavaNengo.
Some of the means of interacting
with a running model in JavaNengo
can be provided using packages
such as IPython and Matplotlib,
but such tools have not been our focus thus far.
It is possible to build a graphical program
like JavaNengo's GUI on top of PyNengo's object model,
but that remains as future work.

To summarize, PyNengo overcomes
JavaNengo's limitations.
It can simulate most models
currently simulated by JavaNengo
using a well-tested codebase of
$\sim$5000 lines of code,
compared to JavaNengo's $\sim$45,000
(which does not include the JavaNengo GUI).
A new internal representation of neural models
allows the OpenCL simulator to simulate
large models on the scale of Spaun
hundreds of times faster than JavaNengo
using inexpensive commodity hardware.
Compatibility with CPython makes scientific Python software
available for preparing and analyzing data.
In all, PyNengo provides a platform for
simulating larger and more complex models than Spaun,
and can therefore further test the NEF
as a theory of neural computation.

\section{Neural Engineering Framework (NEF)}

The Neural Engineering Framework (NEF; \citealp{eliasmith2003})
proposes three quantitatively specified principles
that enable the construction
of large-scale neural models.
Briefly, this mathematical theory defines:
\begin{enumerate}
  \item \textbf{Representation:} A population of neurons
    collectively represents a time-varying vector of real numbers
    through nonlinear encoding and linear decoding.
  \item \textbf{Transformation:} Linear and nonlinear
    functions on those vectors
    are computed by linear decodings
    that are used to analytically compute
    the connections between populations of neurons.
  \item \textbf{Dynamics:} The vectors represented
    by neural populations can be considered state variables
    in a (linear or nonlinear) dynamical system,
    and recurrent connections can be computed
    using principle 2.
\end{enumerate}
Figure~\ref{fig:nef} provides a graphical
summary of these three principles.

\paragraph{Representation}
Information is encoded by populations of neurons.
The NEF represents information
with time-varying vectors of real numbers,
allowing theorists to propose possible
neural computations by
manipulating that information
using conventional mathematics.
The NEF suggests that we can characterize
the \textit{encoding} of
those vectors by injecting
specific amounts of current into
single neuron models based on
the vector being encoded.
This drives the neuron,
causing it to spike.
With enough neurons,
the originally encoded vector
can be estimated
through a \textit{decoding} process.
This idea is a kind of population coding
\citep{georgopoulos1986, salinas1994}, but generalized
to vectors of arbitrary dimensionality.

In the encoding process, the input signal drives
each neuron based on its \textit{tuning curve},
which describes how likely
that neuron is to respond to a given input signal.
The tuning curve is a function of the gain
of a neuron (how quickly the activity rises),
the bias (the activity of a neuron given no signal),
and the \textit{encoding weight}
(the direction in the input vector space
that causes the neuron to be the most active).

In the decoding process,
the spike trains are first filtered
with an exponentially decaying filter
accounting for the process
of a spike generating a postsynaptic current.
Those filtered spike trains are summed together
with weights that are determined
by solving a least-squares minimization problem.
Note that these decoding weights
do not necessarily depend on the input signal;
instead, we typically perform
this minimization on points
sampled from
the vector space
that the population represents.

\paragraph{Transformation}  % NEF as neural model...
Neurons communicate through
unidirectional connections called synapses.
When a neuron spikes,
it releases neurotransmitter across the synapse,
which typically causes some amount of current
to be imparted in the postsynaptic (downstream) neuron.
Many factors affect the
amplitude of the imparted current;
we summarize those factors
in a scalar connection weight
representing the strength
of the connection between two neurons.
In order to compute any function,
we set the connection weights between
two populations to be the product of
the decoding weights for that function
in the first population,
the encoding weights
for the downstream population,
and any linear transform.

This implies that the NEF makes
a hypothesis about synaptic weight matrices;
specifically, that they
have low rank, and can be factored
into encoders, decoders, and a linear transform.
Note that, in practice, we rarely use
the full connection weight matrix,
and instead store
the encoders, decoders, and linear transform separately
(i.e., the three factors of the connection weight matrix).
This provides significant
space and time savings during simulation.

\paragraph{Dynamics}
While feedforward vector transformations
suffice to describe
some neural systems,
many require persistent activity through recurrent connections.
When recurrent connections are introduced,
the vectors represented by neural populations
can be thought of as state variables
in a dynamical system.
The equations governing dynamics
in such a system
can be designed and analyzed
using the methods of control theory,
and translated into neural circuitry
using the principles
of representation and transformation.

\paragraph{NEF and Nengo}
Large models can be built
by using the principles of the NEF
as connectable components
that describe neural systems,
just as a circuit diagram
describes an electronic circuit.
The goal of Nengo is to enable
modellers to create and connect those components.
Components describe
what information is being represented,
and connections describe
how that information is transformed.
Nengo translates those descriptions
to a network of interconnected neurons,
situating it as a ``neural compiler''
that translates
a high-level functional model
to a low-level neural model.

\section{PyNengo object model}

To describe an NEF model,
PyNengo defines six core objects.
\begin{enumerate}
  \item The \texttt{Model} encapsulates the entire model and
    exposes a simplified interface to other objects.
  \item The \texttt{Ensemble} contains a group of neurons
    that encodes a time-varying vector of real numbers.
  \item The \texttt{Node} represents non-neural information,
    such as sensory inputs and motor outputs.
  \item The \texttt{Connection} describes how
    nodes and ensembles are connected.
  \item The \texttt{Network} encapsulates a functionally related
    group of interconnected nodes and ensembles.
  \item The \texttt{Probe} gathers data during a simulation
    for later analysis.
\end{enumerate}
These six objects contain symbolic information
about a Nengo model,
enabling a strict division between
model construction and simulation.
This allows a Nengo model
to be run on multiple simulators.

\subsection{Model}

The \texttt{Model} object is primarily a container
for Nengo objects,
but also doubles as a scripting interface
that simplifies creating, connecting,
and probing objects.
Basic models
can be created solely through interacting
with the \texttt{Model} object
(see sections \ref{sec:comm-channel} \& \ref{sec:lorenz}
for examples).
In advanced models,
objects are instantiated
and then added to the model
(see section \ref{sec:cconv} for an example).

\subsection{Ensemble}

An \texttt{Ensemble} is
a population of neurons
that represents information
in the form of a real-valued vector.
When creating an ensemble,
the user must provide a name,
an object that describes
a population of neurons,
and the dimensionality
(i.e., the length of the vector it represents).
For example,
\begin{quote}
  \texttt{Ensemble('A', nengo.LIF(50, tau\_ref=0.002), 1)}
\end{quote}
describes an ensemble, `A',
that uses 50 leaky integrate-and-fire neurons \citep{lapicque1907}
with a 2 millisecond refractory period
to represent a one-dimensional vector.
The \texttt{nengo.LIF} class defines
the parameters of the LIF neurons symbolically
so that each simulator can compute
the LIF nonlinearity efficiently.
The neuron model used by the ensemble
is changed by passing in a different symbolic neuron object;
however, the simulator used must be aware
of that type of neuron.

Other attributes of the \texttt{Ensemble},
such as its encoding weights,
the maximum firing rate of its neurons,
and so on, can be specified
either as keyword arguments
to the \texttt{Ensemble} constructor,
or by setting an attribute on the instantiated object.
While an ensemble makes a hypothesis
about the information being represented by neurons,
these additional attributes
allow modellers to set
neural parameters according to \textit{in vivo}
electrophysiology data.
If these attributes are not set,
Nengo attempts to maintain
neurobiological constraints
by selecting neural parameters
from distributions consistent with
neocortical pyramidal cells.

\subsection{Node}

A \texttt{Node} contains a user-defined
Python function that directly calculates
the node's outputs from its inputs at each timestep.
Available inputs include
the simulator timestep,
the decoded output of an ensemble,
or the output of another node.
However, unlike ensembles,
there are no constraints on the type
of function that the node computes.
A node can track any number of variables internally,
and use the state of those variables
when computing its function.
For example, it can interact directly with hardware,
and interface with other programs
using shared memory or sockets.

A node represents information
that cannot be decoded from an ensemble.
As a simple example,
a node can be used to model sensory stimuli
that are predefined functions of time.
As a more sophisticated example,
a node can be used to model
a complex experimental environment that
both provides input to the neural model
and responds to the neural model's output.
Nodes allow Nengo to represent
neural components,
the body that those components drive,
and the environment that body interacts with
in a single unified model.
This makes Nengo models more explicit,
and enables simulators
to control and optimize node execution.

\subsection{Connection}

Ensembles and nodes can be connected together
in several ways.
A \texttt{Connection} contains symbolic information
about how two objects are connected.
That information either includes
a factored or full connection weight matrix,
or includes enough information
to generate weights during simulation.

The most important type of connection
in Nengo is the \texttt{DecodedConnection}.
This connection implements
the NEF's transformation principle.
In other words, the \texttt{DecodedConnection}
allows ensembles to project
encoded information---or
a transformation of that information---to
another ensemble or node.
This functionality is what enables Nengo models
to appear conceptual,
even though the underlying implementation
can translate that connection
to synaptic weights.
By default, ensembles are connected
with a \texttt{DecodedConnection}.

However, neurons in an ensemble can be directly connected
to neurons in another ensemble
with synaptic connection weights
using the \texttt{NonlinearityConnection}.
Both connection types
can be temporally filtered,
and the weights involved in the connection
(decoding weights for \texttt{DecodedConnection},
synaptic connection weights for \texttt{NonlinearityConnection})
can be modified over time with learning rules.

\subsection{Network}

A network is a collection of interconnected ensembles and nodes.
It is primarily a way of grouping together
a set of connected objects
that collectively perform a complex function.
Encapsulating them in a network
makes its purpose explicit
and hides the complexity of the function
(see section \ref{sec:cconv} for an example).
\texttt{Network} is a base class designed to be
subclassed by modellers.
The code that creates and connects
several objects in a model can be
grouped into a \texttt{Network} subclass
with only minor changes.

Networks can vary dramatically in size,
and can be composed of other networks.
As a simple example, the \texttt{Integrator} network
is composed of only one recurrently connected ensemble.
By encapsulating that logic in a network,
the purpose of that ensemble is made explicit.
As a complex example,
the \texttt{BasalGanglia} network
is composed of five groups of ensembles
connected with several specific functions
that together implement a ``winner-take-all'' circuit.
Encapsulating the code to create those ensembles
and connections in a network subclass
makes a complicated section of code
easy to include in many different models.

\subsection{Probe}

A \texttt{Probe} monitors
a particular part of another object
in order to record its value throughout a simulation.
Nengo models contain many variables
that change over time,
including membrane potentials,
spike events, and encoded vectors.
It is resource intensive to store the values of
large numbers of variables
at each timestep, and it is also not necessary,
as typically only a small fraction
of these variables are analyzed after a simulation.
The modeller chooses which variables to
record by connecting a probe to an object.

Like nodes, a probe could be implemented
outside of the model.
However, doing so requires detailed knowledge
of the simulator,
and can incur significant overhead
if not implemented carefully.
For these reasons, we have made probes
a core component of a Nengo model,
and are therefore explicit
and optimizable.
Further, since probes are described
at a symbolic level,
the underlying implementation
can output probed data in many different formats.
Currently, simulators store probed data
directly in memory,
but the ability to store data
in files or to stream data
directly to sockets is forthcoming.


\section{PyNengo simulators} \label{sec:simulators}

Decoupling model creation and simulation
has been done previously
by PyNN \citep{davison2008}.
In PyNN, the same Python script
can be used to run a model
on four different simulators.
Nengo follows this programming model by
decoupling neural model creation and simulation,
which enables PyNengo simulators
to allocate memory and schedule computations
in the most efficient ways possible.
Simulators are given a \texttt{Model}
as an argument;
this \texttt{Model} is a symbolic description,
which a simulator can assume will not change.
The simulator, however,
can modify the model as it sees fit;
in most cases, this means that the simulator
will fill in many of the details
not specified at the symbolic level.

We have implemented
a reference simulator using NumPy
for vectorized computations,
and an OpenCL simulator
using PyOpenCL to parallelize
computations on GPUs and multicore CPUs.
In the remainder of this section,
we will describe
the reference simulator implementation;
the OpenCL simulator shares many
of the reference simulator's architectural choices,
but the details of its implementation
include OpenCL-specific optimizations
that are beyond the scope of this paper.
Additionally, it should be noted that
any simulator that takes a \texttt{Model}
as input and exposes a \texttt{step}
function can be used in PyNengo;
the reference simulator
is provided as an example
for simulator designers,
not as a specification.

The reference simulator
uses a reduced set of objects
that describe computations
occurring in the model.
Specifically, the simulator
contains signals, which represent values,
and operators, which represent computations
performed on signals.
Figure~\ref{fig:sim} shows the signals
and operators used in a simple model.

\paragraph{Signals}

A \texttt{Signal} represents any number that
will be used by the simulator.
Each high-level Nengo object contains
several signals;
for example, an ensemble contains signals
that represent the high-level input
signal that will be encoded
to input currents,
and the encoding weights.
It also contains a neural population,
which contains signals that represent
input currents, bias currents,
membrane voltages, and refractory times for each cell.

As can be seen in Figure~\ref{fig:sim},
the signals used in a Nengo simulation
can be conceptually grouped into
those that track low-level neural signals,
and those that track high-level signals
defined by the NEF.
Other neural simulators only track
low-level signals.
Operators commonly map
between related low- and high-level signals.

\paragraph{Operators}

Operators represent computations
to be performed on signals on each timestep.
Once the model has been built,
only a small set of mathematical
operations are necessary for simulation.

Many of the computations
done in a simulation
are linear transformations (e.g.,
the decoding and encoding steps
in Figure~\ref{fig:sim}),
and therefore can share a common operator;
this can be helpful for parallelizing computations.
Nonlinear functions, however,
require specific operators.
Each supported neuron model and learning rule
has an associated operator;
the simulator explicitly maps
from symbolic neuron objects in ensembles
and from symbolic learning rule objects
in connections to their associated operator.

\paragraph{Reference simulator}

Before the first timestep, the reference simulator
\begin{enumerate}
  \item translates high-level objects to
    a set of signals and operators,
  \item allocates NumPy arrays for each signal, and
  \item sorts operators based on a dependency graph.
\end{enumerate}
On each timestep, the reference simulator
\begin{enumerate}
  \item computes each operator in order, and
  \item copies probed signals to memory.
\end{enumerate}
Figure~\ref{fig:sim} depicts
the state of the reference simulator
after two timesteps of a simple model;
all subsequent timesteps perform the same
operations as the first two.

\section{Example scripts} \label{sec:examples}

The scripting interface provides
a simple way to add PyNengo objects to a model,
simulate that model,
and extract data collected during the simulation.
Rather than list the functions in the scripting interface,
we instead provide three concrete example scripts
that highlight the types of models that can be
built with PyNengo.
We have also implemented two of these three examples
in PyNN to provide a comparison
for the length and clarity of the code
describing the models.

\subsection{Communication channel} \label{sec:comm-channel}

A communication channel
is a simple example demonstrating
the representation and transformation principles
of the NEF.
A communication channel
represents some signal and transmits it unchanged;
i.e., it computes the function $f(x) = x$.
Figure~\ref{fig:comm-channel}
depicts a scalar communication channel
representing band-limited Gaussian white noise.

The communication channel
is a simple enough model
that it can be readily implemented in PyNN.
Figure~\ref{fig:pynn} compares the code
for implementing a communication channel in Nengo and PyNN.
This highlights many of the differences
between Nengo models and conventional neural models;
we also use these script for benchmarking
(see section \ref{sec:benchmark}).

\subsection{Lorenz attractor network} \label{sec:lorenz}


Many models in theoretical neuroscience
are based on attractor networks \citep{amit1992, deco2003}.
The NEF has been used in the past
to implement many different types of
attractor networks \citep{eliasmith2005}.
Figure~\ref{fig:lorenz} depicts
a Nengo implementation of the Lorenz chaotic attractor
with a single ensemble
composed of 2000 leaky integrate-and-fire neurons.
We have implemented the Lorenz attractor
in PyNN for benchmarking purposes
(code not shown; the PyNN script is $\sim$100 lines long,
while the Nengo script in Figure~\ref{fig:lorenz}
is 20 lines long).

\subsection{Circular convolution} \label{sec:cconv}


A common function computed by Nengo models
is circular convolution.
This function is a central part
of the Semantic Pointer Architecture \citep{eliasmith2013}
that uses the NEF to build
a variety of cognitive models including Spaun.
Circular convolution
is best implemented in a two-layer network,
rather than in a single connection.
This two-layer organization is simplified in
the \texttt{CircularConvolution} network subclass.
The complexity encapsulated in that network
can be seen in Figure~\ref{fig:cconv}.

Unlike the previous two examples,
we do not implement
circular convolution in PyNN.
The resulting script would be
too long to be instructive.

\section{Benchmarks} \label{sec:benchmark}


While benchmark models are not indicative
of performance on all models,
increasing simulation speed
was a primary motivator in creating PyNengo.
To validate that performance has improved,
we ran the models described in section~\ref{sec:examples}
for various numbers of neurons and dimensions
for each ensemble.

The communication channel and Lorenz attractor
are small models that demonstrate
the principles of the NEF.
Their small size enables us to write
PyNN scripts that implement roughly
the same functionality
with Brian \citep{goodman2009}, NEURON \citep{hines2009},
and NEST \citep{eppler2008}.\footnote{We
  were unable to compile PCSim \citep{pecevski2009} on the
  machine on which we ran benchmarks.}
We ran each parameter set five times
on the same machine,
and plot the mean time elapsed.
In most cases, the coefficient of variation
for the five sample times
is well below 0.1, except for two
outliers with coefficients of 0.18 and 0.22,
overall indicating that the reported means are robust.
The results, shown in Figure~\ref{fig:benchmarks}A and B,
suggest that all versions of Nengo are significantly
faster than the simulators accessible
through PyNN, especially
as the size of models increases.
This is likely due to Nengo's
use of factorized weight matrices,
rather than storing and computing with
the entire weight matrix
on each timestep.
While NEST and NEURON were not
run on multiple cores using message passing,
the reference simulator of PyNengo
also only uses one CPU core.
The results also suggest that PyNengo's
simulators are faster than JavaNengo's simulator.

As a larger-scale example,
we have also benchmarked
the circular convolution model.
Circular convolution is an important test case,
as a significant portion of Spaun's
2.5 million neurons are used to
implement circular convolution.
In this case, only versions of Nengo
were tested.
Instead of running each simulation multiple times,
we instead ran the simulator for 10 timesteps
in order to fill various levels of CPU or GPU cache,
and then ran the simulator for 1000 more timesteps;
there is very little variance using this method.
As can be seen in Figure~\ref{fig:benchmarks}C,
for large models, the OpenCL simulator
performs much faster than JavaNengo;
in particular, a Radeon 7970 GPU performs
500-dimensional circular convolution
with about half a million neurons
faster than real time,
and 50 times faster than JavaNengo.
In the 50-dimensional case,
the Radeon 7970 GPU is 200 times faster
than JavaNengo.
Additionally, although both JavaNengo
and the OpenCL simulator on a CPU
use all available CPU cores,
PyNengo's OpenCL simulator is significantly faster.

\section{Discussion}

\subsection{Comparison to similar projects}

There are many other neural simulators
dedicated to building large-scale neural models;
however, Nengo is unique in being built
on a theoretical framework
that has been validated
through the Spaun model and other past work.

The most closely related project is PyNN
\citep{davison2008},
which provides a high-level scripting
interface similar to the
high-level object model in PyNengo.
Rather than implementing its
own simulator, PyNN uses four existing
simulators with Python bindings,
such as NEST.

The APIs of Nengo and PyNN are similar,
but differ significantly
in how groups of neurons are connected together.
In Nengo, connections commonly describe
the mathematical operation that is performed
through the connection between
two ensembles;
e.g., \texttt{DecodedConnection(A, B,
function=square)} connects ensemble A
to ensemble B, transmitting the square of
the value represented by A to B.
In PyNN, connections commonly describe
features of the connection weight matrix
between two populations;
e.g., \texttt{FixedProbabilityConnector(0.5)}
connects two ensembles together,
with a probability of 0.5
that there will be a connection
between a pair of neurons in the two populations.
This difference reflects the
fundamental difference that Nengo
is built on a theoretical framework
that enables modellers to think
about information processing in the brain
at a conceptual level.

As shown in Figure~\ref{fig:benchmarks},
Nengo is able to simulate many models
faster than each of PyNN's simulators.
This is, in part,
because Nengo stores the factors
of the connection weight matrix,
rather than storing the entire matrix.
However, PyNN is able to simulate
many detailed neuron models
and learning rules,
and has access to the wealth of models
already created for each of its
accessible simulators.
Because Nengo is in an earlier development stage,
many of these detailed neuron models
and learning rules remain to be added.
PyNN is therefore currently better suited to
simulate detailed neural models.

\subsection{Future work}

PyNengo is a young project that
has started with a deliberately minimal base
in order to make development easier than in
previous implementations.
Additionally, many neural modellers
are already using Python,
as evidenced by the existence of
PyNN, Brian, and Python bindings to other simulators.
Nengo will be able to benefit
from improvements to scientific Python tools.
It is also extendable by modellers with
varying technical skills;
creating a new network is simple,
while creating a new simulator is complex, but possible.
By enabling outside contributions,
and providing a user-friendly API
and many example models,
we hope to foster a rich community of modellers
contributing networks, neuron types, learning rules,
and simulators to Nengo.
Internally, our short-term goal
is to implement the JavaNengo use cases
not currently covered by PyNengo.
Our long-term goal is to
create a graphical interface
to build models and
interactively inspect simulations.

\subsection{Conclusion}

PyNengo is the next generation of Nengo.
Though the project is still young,
it can already simulate most models
that have been built using the NEF.
It does this with 11\% as many lines of code
as its predecessor,
and interacts seamlessly with
other scientific Python tools.
While the reference simulator
is simple and easy to understand,
the OpenCL simulator is extremely fast;
it can simulate circular convolution models
50-200 times faster than JavaNengo,
which itself is faster than alternative simulators
on simpler models.
This makes the creation and simulation
of models that are many times larger than Spaun
tractable with current hardware.
These models will further test
the NEF as a theory of neural computation;
PyNengo makes those models
accessible to anyone with
a modern computer.

\section*{Disclosure/Conflict-of-Interest Statement}

The authors declare that the research was conducted in the absence of
any commercial or financial relationships that could be construed as a
potential conflict of interest.

\section*{Author contributions}

TB led development of the
PyNengo object model and scripting interface,
wrote the text of the paper,
and prepared all of the figures.
JB led development of
the PyNengo reference simulator and OpenCL simulator,
edited text, created an early version
of Figure~\ref{fig:sim},
and ran the benchmarks shown in
Figure~\ref{fig:benchmarks}C.
EH contributed significantly
to PyNengo and both of its simulators, and edited text.
TD led development
of a Theano-backed
version of PyNengo that
identified issues with Theano,
and provided the base for
the version of PyNengo described in this paper.
TCS contributed
to PyNengo, helped implement
the PyNN scripts used in
sections~\ref{sec:examples} and \ref{sec:benchmark},
and implemented the JavaNengo scripting interface
on which PyNengo is based.
DR contributed
to PyNengo and the reference simulator,
and edited text.
XC contributed to PyNengo
and the reference simulator.
AV ran the benchmarks
shown in Figure~\ref{fig:benchmarks}A and B.
CE oversaw all development,
contributed to PyNengo,
and co-created the NEF with Charles Anderson.

\section*{Acknowledgments}

We thank Bryan Tripp
for editing this paper,
contributing to PyNengo
and the PyNengo OpenCL simulator
and for creating the JavaNengo simulator
that has been useful
for the past six years,
and will continue
to be used for many years to come.
We thank Peter Blouw and Brent Komer,
who have contributed
to PyNengo by providing examples,
unit tests, and bugfixes.

\paragraph{Funding\textcolon}
NSERC Discovery, NSERC Graduate Fellowships,
NSERC Banting Fellowship, ONR (N000141310419)
and AFOSR (FA8655-13-1-3084).

\bibliographystyle{frontiersinSCNS&ENG}
\bibliography{nengo}

\clearpage

\begin{figure}
 \textbf{\refstepcounter{figure}\label{fig:nef} Figure \arabic{figure}.}{
   Summary of the three principles of the Neural Engineering Framework
   (NEF). \textbf{(A)} By the representation principle, signals are encoded
   in neural populations based on the \textit{tuning curve}
   of each neuron (top). The tuning curve describes
   how active a neuron is given some input signal.
   If we drive the eight neurons in the top panel
   with the input signal in the middle panel,
   we see the spike trains in the bottom panel.
   \textbf{(B)} By the representation principle,
   the spiking activity of a neural population
   can be decoded to recover the original input signal,
   or some transformation of that input signal.
   First, the firing pattern shown in the top panel
   is filtered with a decaying exponential filter (middle panel).
   The filtered activity is then summed together
   with a set of weights that approximates
   the input signal (bottom panel, green)
   and the cosine of the input signal (bottom panel, purple).
   \textbf{(C)} A sine wave is encoded by population A (top panel);
   the negative of that signal is projected
   to population B (middle panel)
   and the square of that signal is projected
   to population C (bottom panel).
   By the transformation principle,
   populations of neurons can send signals
   to another population by decoding
   the desired function from the first population
   and then encoding the decoded estimate
   into the second population.
   These two steps can be combined into a single step
   by calculating a set of weights
   that describe the strength of the connections
   between each neuron in the first population
   and each neuron in the second population.
   \textbf{(D)} A neurally implemented dynamical system
   has negative feedback across its two dimensions,
   resulting in a harmonic oscillator.
   The oscillator is plotted across time (top)
   and in state space (bottom).
   By the dynamcs principle,
   signals being represented by population of neurons
   can be thought of as state variables in a dynamical system.
   }
\end{figure}

\begin{figure}
 \textbf{\refstepcounter{figure}\label{fig:sim} Figure \arabic{figure}.}{
   Detailed breakdown of the PyNengo reference simulator
   running a simple model for two timesteps.
   \textbf{(A)} Code describing the model being simulated.
   It consists of ensemble A projecting the sine of its
   encoded signal to ensemble B,
   which is recurrently connected.
   \textbf{(B)} Diagram depicting the model being simulated.
   \textbf{(C)} A detailed diagram of how the reference simulator
   organizes this model. Signals (blue) represent the values
   tracked in the simulation.
   Operators (red) represent the computations done on signals.
   Signals can be grouped as low-level neural signals
   that are used to compute the nonlinear functions
   underlying neuron models,
   and high-level NEF signals that are used to
   drive neurons and track the signals
   that the neurons are representing.
   The operators that implement the decoding
   and encoding steps map between
   the low-level neural signals
   and the high-level NEF signals.
   \textbf{(D)} The signals tracked at the low level
   can be interpreted as a model
   commonly seen in computational neuroscience literature;
   a population of leaky integrate-and-fire neurons
   is driven by some time-varying input current, $J(t)$.
   These neurons project to a population
   of recurrently connected neurons.
   The connection weights between the two populations,
   and from the second population to itself,
   can be computed by the NEF's transformation
   principle, bypassing the need for
   the high-level NEF signals
   used by the reference simulator
   for speed and data collection purposes.
   \textbf{(E)} The signals tracked at the high level
   can be interpreted as a dynamical system.
   State variable A simply represents its input,
   and passes its state to a sine function
   which becomes the input to B.
   State variable B is a simple linear system
   that can be described with the typical
   $\dot{x}(t) = A x(t) + B u(t)$ equation.
   These dynamical systems can be simulated
   directly, without the use of spiking neurons,
   in order to quickly analyze system behavior,
   if desired.}
\end{figure}

\begin{figure}
 \textbf{\refstepcounter{figure}\label{fig:comm-channel}
   Figure \arabic{figure}.}{
   A communication channel implemented with Nengo.
   \textbf{(A)} Diagram depicting the model. Ensemble A
   projects its encoded signal to ensemble B unchanged.
   \textbf{(B)} Nengo code to build and simulate the model
   for 1 second.
   \textbf{(C)} The results of the simulation.
   The input signal (top panel) is white noise limited to 0 to 5 Hz.
   The signal is well represented by both ensemble A (middle panel)
   and ensemble B (bottom panel) despite the neural firing patterns
   (underlaid in middle and bottom panels) being different.}
\end{figure}


\begin{figure}
 \textbf{\refstepcounter{figure}\label{fig:pynn} Figure
   \arabic{figure}.}{
   Implementation of the communication channel (left) in Nengo
   and (right) in PyNN. Solving for decoding weights
   takes approximately 40 lines of code,
   which are not included in this figure.}
\end{figure}

\begin{figure}
 \textbf{\refstepcounter{figure}\label{fig:lorenz} Figure \arabic{figure}.}{
   A Lorenz attractor implemented with Nengo.
   \textbf{(A)} Nengo code to build and simulate the model
   for 6 seconds.
   \textbf{(B)} Diagram depicting the model. The state ensemble
   is recurrently connected with a complex function
   implementing the dynamics of the Lorenz attractor.
   Note that this populations does not receive
   any input that might drive its initial value;
   instead, the initial value is determined by
   the baseline firing of the 2000 leaky integrate-and-fire
   neurons that make up the state ensemble.
   \textbf{(C)} The trajectory that the state ensemble takes
   in its three-dimensional state space.
   For the parameters chosen, the trajectory takes
   the well-known butterfly shape.
   \textbf{(D)} The state vector plotted over time.
   \textbf{(E)} The spikes emitted by a random sample of 25
   neurons from the state ensemble.
   Some neurons fire uniformly across the 6 second simulation,
   but most change depending on the state being tracked
   due to the recurrent connection.}
\end{figure}

\begin{figure}
 \textbf{\refstepcounter{figure}\label{fig:cconv} Figure
   \arabic{figure}.}{
   Circular convolution implemented with Nengo.
   \textbf{(A)} Nengo code to build and simulate the model
   for 0.2 seconds.
   \textbf{(B)} Diagram depicting the model.
   The input vectors, A and B, represent four-dimensional vectors
   which are mapped onto six ensembles within the
   Circular Convolution network through
   complicated transformation matrices
   that implement a discrete Fourier transform.
   Each ensemble within the network represents a
   two-dimensional vector. The product of the two dimensions
   is projected through another complicated transformation matrix
   that implements the inverse discrete Fourier transform,
   computing the final four-dimensional result.
   Note that the complicated parts of the model
   are contained within the network;
   the number of ensembles and the transform matrices shown
   are automatically generated by the network depending on
   the dimensionality of the input vectors.
   \textbf{(C)} The result of the simulation.
   Straight horizontal lines represent
   the target values that each ensemble
   should represent.
   Wavy lines represent the decoded values
   for each dimension represented by the
   A, B, and Result ensembles
   (top, middle and bottom panels, respectively).
   The ensembles represent the correct values,
   after a startup transient of less than 0.1 seconds.}
\end{figure}

\begin{figure}
 \textbf{
   \refstepcounter{figure}\label{fig:benchmarks}
   Figure \arabic{figure}.}{
   Benchmark results for several simulators
   on the example models described
   in section~\ref{sec:examples}.
   In (A) and (B), all of the simulators
   except the PyNengo OpenCL simulator
   were run on an Intel Core i7-965.
   JavaNengo used all 4 cores of this processor;
   all other simulator used only 1 core.
   The PyNengo OpenCL simulator
   was run on an NVidia GTX280 GPU.
   \textbf{(A)} Benchmark results from simulating
   the communication channel for 10 simulated seconds
   at a 1 millisecond timestep.
   For all model sizes, PyNengo simulators
   are faster than JavaNengo,
   which is significantly faster than NEURON and NEST,
   which are significantly faster than Brian.
   The full Brian results are not shown;
   for the largest model, Brian takes $\sim$768 seconds.
   \textbf{(B)} Benchmark results from simulating
   the Lorenz attractor for 10 simulated seconds
   at a 1 millisecond timestep.
   For most model sizes,
   the results are the same as (A),
   except that NEURON is notably faster.
   The full results for Brian and NEST are not shown;
   for the largest model, they take $\sim$1467
   and $\sim$601 seconds respectively.
   \textbf{(C)} Benchmark results from simulating
   circular convolution for 1 simulated second
   at a 1 millisecond timestep.
   For the blue lines, the simulator used
   was the PyNengo OpenCL simulator.
   The CPU used for JavaNengo and
   the PyNengo reference simulator
   was an Intel Core i7-3770;
   all 4 cores were used by JavaNengo,
   while PyNengo's reference simulator
   only used one core.
   For all model sizes,
   the OpenCL simulator is faster
   than the JavaNengo simulator,
   which is faster than the PyNengo
   reference simulator.
   The reference simulator was only run
   up to 50 dimensions.
   The full results for JavaNengo
   are not shown; for the largest model,
   it takes $\sim$45 seconds.
 }
\end{figure}

\end{document}
