\documentclass{frontiersSCNS}

\usepackage{url,lineno}
% \linenumbers  % TODO uncomment!

\graphicspath{{figures/}}

\copyrightyear{2013}
\pubyear{2013}

\def\journal{Neuroinformatics}
\def\DOI{}
\def\articleType{Methods Article}
\def\keyFont{\fontsize{8}{11}\helveticabold }
\def\firstAuthorLast{Bekolay {et~al.}}
\def\Authors{Trevor Bekolay\,$^{1,*}$, James Bergstra\,$^{1}$,
  Xuan Choo\,$^{1}$, Travis DeWolf\,$^{1}$, Eric Hunsberger\,$^{1}$,
  Daniel Rasmussen\,$^{1}$, Terrence C. Stewart\,$^{1}$
  and Chris Eliasmith\,$^{1}$}
\def\Address{$^{1}$Computational Neuroscience Research Group,
  Center for Theoretical Neuroscience,
  University of Waterloo, Waterloo, ON, Canada}
% The Corresponding Author should be marked with an asterisk
% Provide the exact contact address (this time including street name
% and city zip code) and email of the corresponding author
\def\corrAuthor{Trevor Bekolay}
\def\corrAddress{David R. Cheriton School of Computer Science,
  University of Waterloo, 200 University Avenue West,
  Waterloo, ON, N2L 3G1, Canada}
\def\corrEmail{tbekolay@uwaterloo.ca}


\begin{document}
\onecolumn
\firstpage{1}

\title[Nengo in Python]{Reimplementing Nengo in Python for speed and simplicity}
\author[\firstAuthorLast ]{\Authors}
\address{}
\correspondance{}
\extraAuth{}
%\extraAuth{corresponding Author2 \\ Laboratory X2, Institute X2, Department X2,
%  Organization X2, Street X2, City X2 , State XX2
%  (only USA, Canada and Australia), Zip Code2, X2 Country X2, email2@uni2.edu}
\topic{Python in Neuroscience II}

\maketitle


\begin{abstract}
%Maximum: 2000 characters
  Nengo is a software package for creating and simulating
  large-scale neural models built with
  the Neural Engineering Framework
  and the Semantic Pointer Architecture.
  Nengo was recently used to build Spaun, the largest
  functional brain model to date.
  However, this version of Nengo simulated Spaun
  9000 times slower than real time,
  and required modellers to know
  both Java and Python to be productive.
  In this paper, we describe a Python reimplementation of Nengo
  that leverages scientific Python packages
  to produce the same large-scale neural models
  with an order of magnitude less code.
  The model definition and simulation steps
  have been decoupled,
  making it possible to run the same model
  on different software simulators
  and different hardware platforms.
  Two software simulators are implemented,
  including an OpenCL simulator
  that is TODO times faster than
  the previous version of Nengo
  for a model computing the circular convolution
  of TODO dimensional vectors.

  \tiny \keyFont{\section{Keywords:} 1 2 3 4 5 (6 7 8)}
\end{abstract}

%max figures+tables: 15
%max legnth: 12000 words
%max pdf length: 12 pages

\section{Introduction}

Nengo is a set of software tools built around
a theoretical framework for building neural models
proposed by Eliasmith \& Anderson \cite{TODO},
and more recently, an architecture
for performing high-level cognitive tasks
in that framework proposed by Eliasmith \cite{TODO}.
Nengo consists of a Java-based simulator,
a Jython-based scripting interface
for common use cases of the simulator,
and a Java-based GUI that
uses both the underlying simulator
and the scripting interface.
This set of tools has been used
to create models of visual attention \cite{TODO},
inductive reasoning \cite{TODO},
and reinforcement learning \cite{TODO},
as well as Spaun, a cognitive agent able
to do these and several other tasks
(\cite{TODO}).
Nengo has been used by dozens of researchers
in labs at the University of Waterloo,
Johns Hopkins University, Yale, Stanford,
and others (TODO: confirm).

Here, we describe a new tool
for creating and simulating NEF-based models
implemented in Python,
which effectively merges two existing tools
into one unified package
that can be extended in
several ways that previous Nengo tools could not.

\subsection{Why reimplement Nengo?}

The original Java simulator was created
with the intention of being a general
neural simulator that included
the methods of the Neural Engineering Framework
(NEF; \cite{TODO}) as one of many different
methods of creating neural models.
However, since creating the simulator in 2007,
it has only been extended minimally,
but has been used extensively for its ability
to create NEF-based models.

The Jython scripting interface,
described in \cite{TODO},
provided a simpler interface
to the NEF-specific portions of the simulator.
However, the Java GUI was developed at the same time
as the Jython scripting interface,
and for that reason primarily
interacts with the simulator directly.
Since large-scale models often require
a scripting language for looping
and parameterization,
this has resulted in a difficult transition
for modellers moving from the
drag-and-drop graphical interface
to the scripting interface.
Additionally, advanced scripts often
contain Jython code that either defines
or manipulates Java objects directly,
meaning that modellers have to know
a certain amount of Python and Java to be productive.

Since Nengo makes it possible to define
very large models,
performance is an important issue.
While Java has good support for threads,
it is difficult to interface with
platform-specific tools such as
graphical processing units (GPUs),
which can quickly perform the types of parallel
computations that these neural models require.
This is due to the
architecture of the Java simulator
and the Java Native Interface,
which results in code that
is difficult to maintain and debug.

The new implementation of Nengo described in this paper
was undertaken with three goals in mind:
\begin{enumerate}
  \item Provide a scripting interface
    similar to the existing Jython interface.
  \item Leverage existing scientific computing tools
    rather than write our own.
  \item Provide the ability to run models with
    different software and hardware simulators.
\end{enumerate}

In order to achieve these goals,
this new version of Nengo uses the Python language
and targets the CPython interpreter,
allowing Nengo to use existing scientific Python
tools for vectorized computations, plotting, and graph manipulation.
As evidence that we have met these goals,
we have implemented two different software simulators,
and have done so with an order of magnitude
less lines of code than the Java simulator
and Jython scripting interface combined.
For the remainder of the paper,
we will use ``Nengo'' to refer to
this new Python rewrite.

\subsection{Neural Engineering Framework (NEF)}

TODO; make sure it meshes with 3.1's operators

\begin{equation} \label{TODO}
  TODO
\end{equation}

%% This should be a paper of its own.

%% \subsection{Semantic Pointer Architecture (SPA)}

%% TODO; should we even include this section?
%% It could probably be a separate paper on its own.

\section{Model definition}

One of the strengths of the NEF
is that it allows neural modellers to think
at a higher level
than is normally associated with neural models.
With the NEF, a modeller decides
what information is being represented
by populations of neurons and how that information
is being transformed through the connections
between populations;
this contrasts with modelling decisions
that are typically thought to be important,
such as the morphology of each neuron,
and the distribution of connection weights
between populations.
Instead, these decisions are made---often
automatically---as a direct result
of the information
being represented and the transformations
being performed on that information.
This situates Nengo as a
``neural compiler'' that can
take a functional model description
and produce a low-level neural model
that can be simulated
with any neural simulation software.

To emphasize Nengo's role as a neural compiler,
Nengo allows models to be defined
in a simulator-agnostic manner,
allowing for functionally identical models to be run
on different software and hardware platforms.

\subsection{Nengo objects} \label{sec:objects}

\begin{figure}
\begin{center}
  Nengo high-level object hierarchy
\end{center}
 \textbf{\refstepcounter{figure}\label{fig:01} Figure \arabic{figure}.}{
   Nengo high-level object hierarchy}
\end{figure}

A Nengo model is a collection
of Nengo objects connected together,
and probed in order to collect data
during simulation.
The \texttt{Model} object is primarily a container
for Nengo objects,
but also provides methods
that simplify creating, connecting,
and probing these objects.
Basic Nengo models
can be created solely through interacting
with the \texttt{Model} object.
For more advanced models,
these objects can be instantiated
directly and then added to the model.

There are two types of objects
defined in the main scripting interface.

\paragraph{Ensembles}
The most common is the \texttt{Ensemble},
which is a population of neurons
that represents information
in the form of a real-valued vector.
As described in \eqref{TODO},
this can be done
with any neuron model, $G$.
An ensemble must be given as arguments
its dimensionality
(i.e., the length of the vector it represents),
and an object that describes
the population of neurons.
For example, \texttt{LIF(50, tau\_rc=0.02)}
describes a population
of 50 leaky integrate-and-fire neurons
with an RC time constant of 0.02 seconds.
Neurons are defined in this symbolic manner
so that each simulator can compute
this population's nonlinear function
however it sees fit.

Other attributes of the \texttt{Ensemble},
such as its encoding weights, $\mathbf{e_i}$,
the maximum firing rate of the neurons,
and so on, can also be specified.
This is usually done according
to what is known about
the neurobiology of the area being modelled;
some areas of the brain regularly spike
at rates above 200Hz,
while others rarely spike above 10Hz.
If these attributes are not set,
then they will be randomly selected
from distributions that describe
what is known about pyramidal cells in neocortex.

%% However, not all simulators can support all neuron types,
%% and all of the attributes that can be
%% set on an ensemble.
%% This is particularly the case
%% when interacting with special-purpose hardware.
%% In these cases, simulators should
%% raise appropriate exceptions
%% when unsupported ensembles are encountered.

\paragraph{Nodes}
A \texttt{Node} tracks any information
that is not represented by an ensemble.
In the most common case,
nodes provide input signals
that drive ensembles of sensory neurons.
In a more general sense,
nodes represent the experimental environment
in which a neural model exists.
More practically,
nodes can be used to directly compute
parts of a neural model
for debugging purposes.

A node computes an arbitrary function
of its inputs directly.
Possible input signals include
the simulator timestep,
the decoded output of an ensemble,
or the output of another node.
However, unlike ensembles,
there are no constraints on the type
of function that the node computes.
A node can track any number of variables internally,
and use the state of those variables
when computing its function.
It can interact directly with hardware,
interface with other programs
using shared memory or sockets,
and so on.

Nodes allows many of the special-purpose
routines that are necessary in many real-world
models to be integrated
as core components of a Nengo model.

\paragraph{Connections}

Ensembles and nodes can be connected together
in several ways.
Like other high-level objects,
a connection contains symbolic information
about how two objects are to be connected
together.

The most important type of connection
in Nengo is the \texttt{DecodedConnection}.
This connection implements
the transformation principle equations,
\eqref{TODO}.
In other words, the \texttt{DecodedConnection}
allows ensembles to project
the information they're representing---or
a transformation of that information---to
another ensemble or node.
This functionality is what enables Nengo models
to appear conceptual,
even though the underlying implementation
still connects populations of neurons
together with connection weights.
%TODO: reword? is it important that it's a factored weight matrix

\paragraph{Probes}

Ideally, the entire state of a simulation
would be recorded on each timestep,
and stored with no overhead to the simulator.
In practice, recording data
is a costly part of most simulations,
and often what is simulated must also change
if the signals being probed must be filtered.
For this reason, probes are also
a core component of a Nengo model.

In general, Nengo objects and connections
each have several signals that can be probed.
For example, an ensemble's decoded value,
or the underlying neurons' membrane voltages
can be probed, among other signals.
The underlying logic for the probe
leverages connection logic,
making it easy to probe any part of an object.

Since probes are also described
at a symbolic level,
the underlying implementation
can output probed data in many different formats.
Currently, simulators store probed data
directly in memory,
but adding the ability to store data
in files or to stream probed data
directly to sockets will be added.

\section{Model simulation} \label{sec:simulators}

Once a model has been fully defined,
a simulator can be created to
run that model.
Simulators can be created
that work directly with the
objects described in section \ref{sec:objects}.
However, in the simulators that we have implemented
as reference, the model goes through a
two-stage build process in order to
simulate models as fast as possible.
In the first stage,
the model is ``built;''
i.e., the objects in the model are mapped to a simpler
set of objects that represent the computations
that the simulator must perform.
Those computations are analyzed in order to
schedule them efficiently.
In the second stage,
the simulator performs the computations
that have been scheduled on each timestep.

This two-stage process has enabled us
to implement a more efficient simulator
that is only aware of
the objects that are created
during the build process.
In the future,
new simulators can target either
the high level Nengo objects,
or the low level simulator objects
depending on the constraints of the platform.

\subsection{Reference build process}

\begin{figure}
\begin{center}
  Reference simulator build process
\end{center}
 \textbf{\refstepcounter{figure}\label{fig:01} Figure \arabic{figure}.}{
   Reference simulator build process}
\end{figure}

The first stage of the reference implementation
maps all of the Nengo objects
down to a set of signals and operators.

\paragraph{Signals}
A \texttt{Signal} represents any number that
will be used by the simulator.
Each high level Nengo object contains
several signals;
for example, an ensemble contains signals
that represent the high-level input
signal that will be encoded
to input currents (see equation \eqref{TODO}),
and the encoding weights
($\mathbf{e_i}$; see equation \eqref{TODO}).
It also contains a neural population,
which contains signals that represent
input currents
($J(t)$; see equation \eqref{TODO}),
bias currents ($J^{bias}$; \eqref{TODO}),
membrane voltages ($V$; \eqref{TODO}),
and refractory times for each cell.
These signal are symbolic,
and contain only information
about the size of the signal that will be represented,
and optionally an initial value.
Simulators may therefore represent
these signals however they want internally.

In some cases, determining
the initial values of some signals
may involve some computation.
In particular, solving for decoding weights
(equation \eqref{TODO}) requires
solving a least-squares minimization problem.
This computation is done in this stage
of the build process,
as it is computationally intensive,
and may be implemented
differently by some simulators.

An important feature of signals
is that parts of a signal
can be accessed with Python slice syntax,
which results in a view of the original signal.
For example, when connecting to
a subset of a population of neurons,
current would be injected
in a view of the overall input current signal
for that population,
rather than creating a new signal,
or partitioning the population's signal
into smaller chunks.
Grouping related signals together
and exposing them through views
also provides hints to the simulator
that can be used to organize signals
efficiently in memory.

\paragraph{Operators}
Operators represent computations
to be performed on signals on each timestep.
Once the model has been built,
only a small set of mathematical
operations are necessary for simulation.
\begin{enumerate}
  \item \textbf{Reset} assigns a constant value to a signal.
    This is typically used to set a signal to 0
    at the start of a timestep.
  \item \textbf{Copy} assigns the value
    of one signal to another signal.
    This is typically used instead of \textbf{Reset}
    in situations where the signal's value
    at the start of a timestep should
    be the value of another signal.
  \item \textbf{DotInc} increments a signal by
    the dot product of two signals;
    i.e., $Y \leftarrow Y + AB$.
    This is a common operation
    done in many situations,
    such as decoding a signal (see equation \eqref{TODO}).
  \item \textbf{ProdUpdate} scales a signal
    and adds the dot product of two signals to it;
    i.e., $Y \leftarrow BY + AX$.
    This is done when temporally filtering a signal.
  \item \textbf{Nonlinearities} include any
    user-defined function computed by a node,
    and the update equation for each neuron model.
\end{enumerate}

It is important to note that all operators
excepts for the nonlinearities are linear,
and therefore can share an implementation
in an underlying simulator,
allowing those operations to happen in parallel.

Operators are created
by Nengo objects during the build process.
For example, an ensemble creates
a \textbf{Reset} operator to reset
its input signal to 0,
and a \textbf{DotInc} operator
to encode that input signal into
input currents injected into
the underlying population of neurons;
i.e., the \textbf{DotInc} computes
$J = \alpha_i \mathbf{e_i} X$ from equation \eqref{TODO}.

\subsection{Reference simulator}

\begin{figure}
\begin{center}
  Reference simulator simulation (include dependency graph)
\end{center}
 \textbf{\refstepcounter{figure}\label{fig:01} Figure \arabic{figure}.}{
   Reference simulator simulation}
\end{figure}

The reference simulator starts by
building NumPy arrays
for every signal created during the build process
described above.
It then builds a graph of the
operators created during the build process,
ordering them such that they can be
executed one after another to get the correct answer.

The ordering is determined
by which signals an operator
sets, increments, reads, and updates.
A signal can only be set or updated once,
but can be read or incremented any number of times.
An update represents the operator that sets
the signal's value at time $t+dt$,
and therefore must be the last operator
that acts on a given signal.
This means that an operator
that updates a signal depends on
all operators that set, increment,
or read that signal.
Reads represent reading the value
of a signal at time $t$,
which is determined by the sets
and increments, and so
an operator that reads a given signal
depends on all operators that
set and increment that signal.
Increments must happen
after a given signal is set,
so an operator that increments a given
signal depends on the operator
that sets that signal, if it exists.

The reference simulator represents
these dependencies by a directed graph,
with edges going from
dependent operators to
the operators on which they depend.
The dependency graph is then topologically sorted,
and that ordering is stored.

The simulator can then be run
for an arbitrary amount of time.
On each timestep,
the simulator goes through each of the
operators in order,
and then makes a copy of any signals
that were probed on this timestep.

\subsection{OpenCL simulator}

OpenCL is a framework for writing
efficient parallel code that runs
on many different platforms,
including modern CPUs and GPUs.
It allows for the kind of parallel execution
that is required to run neural models quickly,
while still being able to run
on multiple platforms.

The OpenCL simulator operates on
the same objects that the reference simulator does;
i.e., it starts with a collection
of signals and operators,
schedules them efficiently,
and runs all operators on each timestep.

However, in order to execute
many operators in parallel,
the OpenCL simulator
organizes computations
differently than the reference simulator.
All of the linear operators are mapped
onto \textbf{MultiProdUpdate} operators,
which perform the general update
$$Y \leftarrow \gamma + \beta Y + AX.$$
It can then schedule many
\textbf{MultiProdUpdate} operations
on a single timestep,
if the dependency graph allows it.

The OpenCL simulator also transparently
attempts to build OpenCL equivalents
for arbitrary Python functions
executed by nodes.
These arbitrary functions
can slow down a simulation significantly,
as data and control has to be passed back
to the Python interpreter
on each timestep.
The OpenCL simulator inspects
the abstract syntax tree
of the function being computed
and substitutes OpenCL equivalents when possible.

\section{Extensibility}

Nengo is designed to be extended
in order to make modelling easier,
and to enable models to be run
on many different platforms.

\subsection{Models}

At model creation time,
everything must be done
in a simulator agnostic manner.
Since ensembles and nodes
are the only objects that exist,
models can become long scripts
that contain many
similar function calls that
create and connect ensembles and nodes.
In order to make modelling scripts
shorter and more expressive,
we have created the \texttt{Network} abstraction.

\paragraph{Networks}
A network is a collection of ensembles and nodes
connected together in a particular way.
It is primarily a way of grouping together
a common set of objects connected in a particular way,
and provides some implicit expressiveness
in the name of the network.

Networks can vary dramatically in size,
and can be composed of other networks.
The \texttt{Integrator} network, for example,
is composed of only one recurrently connected ensemble.
By encapsulating that logic in a network,
the purpose of that ensemble is made explicit.
On the other end of the size spectrum,
the \texttt{BasalGanglia} network,
is composed of five groups of ensembles
connected with several specific functions
that together implement a ``winner take all'' circuit.
Encapsulating that logic
makes this complicated section of code
easy to include in many different models
(see Figure~\ref{TODO}).

\begin{figure}
\begin{center}
  Network code listing
\end{center}
 \textbf{\refstepcounter{figure}\label{fig:01} Figure \arabic{figure}.}{
   Network code listing}
\end{figure}

\subsection{Nonlinearities}

Many modellers make use of
different neuron models and
learning rules than are
currently implemented
in the two simulators
in section \ref{sec:simulators}.
Because Nengo endeavors
to compute nonlinear functions efficiently,
these nonlinearities
must be implemented in each simulator,
though they may share the same representation
at the model definition stage.
While this results in simulators
that cannot run all Nengo models,
this cannot be avoided,
and therefore we make this explicit.

In order to implement
a new neuron model or learning rule,
two classes must be created.
The first is a symbolic class
that keeps track of any parameters
that might affect the execution
of the nonlinearity;
for example, for
the leaky integrate-and-fire neuron model,
the refractory time constant,
$\tau^{ref}$, is part of this symbolic class.
An instance of this class
is then used as an argument
to the ensemble constructor.
For learning rules,
an instance of that class
is used as an argument
to functions that connect ensembles.
The second class to be created
is simulator-specific,
and contains the function that will be
executed on each timestep of the simulation.
Individual simulators may have additional
requirements when adding
a new nonlinearity.

\subsection{Model simulation}

The simulator included with Nengo
is provided primarily as a reference
for platform experts designing a simulator
for their platform.
The reference simulator is implemented with
a two-step build process;
this is designed to make
some simulators easier to implement,
as the result of
the first step of the build process
is a set of computations
that most computing devices are able to perform.

However, other simulators---especially
those running on special purpose
hardware---may not be able
to perform all of these computations.
In that case, these simulators
can define their own build process
that takes as input a Nengo model.
This flexibility is possible
due to the decoupling of
the model definition and simulation steps.

\section{Examples and benchmarks}

\subsection{Lorenz attractor network}

Many models in theoretical neuroscience
are based on attractor networks.
The NEF has been used in the past
to implement many different types of
attractor networks \cite{TODO}.
In this example,
we implement the Lorenz chaotic attractor
with a single ensemble
composed of 2000 leaky integrate-and-fire neurons.

\begin{figure}
\begin{center}
  Nengo Lorenz code listing
\end{center}
 \textbf{\refstepcounter{figure}\label{fig:01} Figure \arabic{figure}.}{
   Nengo Lorenz code listing}
\end{figure}

In order to compare this Nengo model
with other pieces of neural modelling software,
we have implemented the Lorenz attractor
in PyNN \cite{TODO} in order to simulate it
with the simulators that PyNN supports.

\begin{figure}
\begin{center}
  PyNN Lorenz code listing
\end{center}
 \textbf{\refstepcounter{figure}\label{fig:01} Figure \arabic{figure}.}{
   PyNN Lorenz code listing}
\end{figure}

\begin{figure}
\begin{center}
  Lorenz attractor benchmarks
\end{center}
 \textbf{\refstepcounter{figure}\label{fig:01} Figure \arabic{figure}.}{
   Lorenz benchmarks}
\end{figure}

\subsection{Circular convolution}

One of the most common operations done
in Nengo models is circular convolution.
Circular convolution is how
the semantic pointer architecture
(SPA; \cite{TODO})
binds two semantic pointers together.
Binding is central to many
of the high level cognitive tasks
that make Spaun unique.
Circular convolution
is also best implemented in a two-layer network,
rather than in a single connection.
This two-layer network construction is simplified
by the existence of the \texttt{CircularConvolution} network.

\begin{figure}
\begin{center}
  Nengo CConv code listing
\end{center}
 \textbf{\refstepcounter{figure}\label{fig:01} Figure \arabic{figure}.}{
   Nengo CConv code listing}
\end{figure}

TODO: can we do circconv in PyNN?

\begin{figure}
\begin{center}
  PyNN CConv code listing
\end{center}
 \textbf{\refstepcounter{figure}\label{fig:01} Figure \arabic{figure}.}{
   PyNN CConv code listing}
\end{figure}

\begin{figure}
\begin{center}
  CConv benchmarks
\end{center}
 \textbf{\refstepcounter{figure}\label{fig:01} Figure \arabic{figure}.}{
   CConv benchmarks}
\end{figure}

\section{Discussion}

Even at its \texttt{0.1} release,
we have met our goals
for this reimplementation of Nengo.
\begin{enumerate}
  \item We have provided a scripting interface
    similar to the Jython interface.
    In many cases, the new scripting interface
    is simpler and more expressive than
    the previous Jython interface.
  \item We have leveraged existing scientific computing tools
    rather than writing our own.
    In the the scripting interface and the reference simulator,
    we use NumPy to simplify array computations.
    In the reference simulator, we use
    NetworkX to perform the topological sort
    of the operator dependency graph.
    In examples, we use Matplotlib
    to plot the results of simulations.
  \item We have provided the ability to run models with
    different software and hardware simulators.
    We have shown that this is possible by
    implementing two software simulators.
\end{enumerate}

While this reimplementation of Nengo
does not completely replace
the existing suite of Java-based Nengo tools,
it is able to cover the vast majority of use cases
with an order of magnitude less
code than the Java-based tools
($\sim$30,000 lines of Java + 10,000 lines of Jython
vs. $\sim$8,000 lines of Python).
We are optimistic that the remaining use cases
can be implemented in this new framework.

TODO: something about documentation, unit tests,
coverage, travis-ci, etc

\subsection{Comparison to similar projects}

There are many other neural simulators
dedicated to building large-scale neural models;
however, Nengo is unique in being built
on a theoretical framework (the NEF, \cite{TODO})
that has already been used to make
large-scale functional brain models.

The most closely related project is PyNN
\cite{TODO},
which provides a high-level scripting
interface similar to the
model creation part of Nengo.
Rather than implementing its
own simulator, PyNN uses existing
simulators, such as Brian \cite{TODO},
NEST \cite{TODO}, and NEURON \cite{TODO}.

The APIs of Nengo and PyNN are similar,
but differ significantly
in how groups of neurons are connected together.
In Nengo, connections commonly describe
the mathematical operation that is performed
through the connection between
two ensembles;
e.g., \texttt{DecodedConnection(A, B,
function=product)} connects ensemble A
to ensemble B, projecting the product of
some dimensions encoded by A to B.
In PyNN, connections commonly describe
features of the connection weight matrix
between two populations;
e.g., \texttt{FixedProbabilityConnector(0.5)}
connects two ensembles together,
with a probability of 0.5
that there will be a connection
between a pair of neurons in the two populations.

As shown in Figure~\ref{TODO},
Nengo is able to simulate many networks
faster than all of PyNN's simulators
(TODO: right?).
This is, in part, because Nengo stores two vectors
whose product is the entire
connection weight matrix between
two ensembles, rather than storing
the entire matrix.

However, PyNN also contains functionality
not currently implemented in Nengo.
PyNN can assign spatial information
to neurons in a population,
which can be used to influence
how those neurons connect to other populations.
PyNN simulates many different learning rules
and neuron types,
and is therefore better suited to
simulate small networks of detailed neuron models
(as would be necessary for modelling
an invertebrate nervous system).

\subsection{Future work}

Nengo is a young project that
has started with a deliberately minimal base
in order to make contributing easier than in
previous implementations.
Additionally, many neural modellers
are already using Python,
as evidenced by the existence of
PyNN, Brian, and Python bindings to other simulators.
Nengo will be able to benefit
from improvements to scientific Python tools.
It is also extendable by modellers with
varying technical skills;
creating a new network is simple,
while creating a new simulator is complex.

By enabling outside contributions,
and providing a user friendly API
and many example models,
we hope to foster a rich community of modellers
contributing networks, nonlinearities,
and simulators to Nengo
and to theoretical neuroscience
as a whole.

\section*{Disclosure/Conflict-of-Interest Statement}

The authors declare that the research was conducted in the absence of
any commercial or financial relationships that could be construed as a
potential conflict of interest.

\section*{Acknowledgement}

We would like to thank
everyone who has contributed
to this reimplementation of Nengo
by providing examples,
unit tests, and bugfixes:
Peter Blouw, Brent Komer, and Bryan Tripp.

\paragraph{Funding\textcolon}
NSERC CRC, CFI, and OIT.

\bibliographystyle{frontiersinSCNS&ENG}
\bibliography{nengo}

\end{document}
