\documentclass[12pt]{article}

\usepackage[english]{babel}
\usepackage[utf8x]{inputenc}
\usepackage{amsmath}
\usepackage{graphicx}
\usepackage{fullpage}
\usepackage{multicol}
\usepackage{natbib}
\usepackage[textwidth=0.8in, shadow]{todonotes}

\setlength{\bibsep}{0.0pt}

% \title{Reimplementing Nengo in Python for speed and simplicity}

\begin{document}

\thispagestyle{empty} % Suppress page number

\begin{centering}
{\large \bf Reimplementing Nengo in Python for speed and simplicity}

\vspace{12pt}
{Trevor Bekolay, James Bergstra, Xuan Choo, Travis DeWolf,
Eric Hunsberger, \\ Daniel Rasmussen, Terrence Stewart, Chris Eliasmith}

{\it Centre for Theoretical Neuroscience, University of Waterloo}

\vspace{12pt}
\end{centering}

  %Nengo has recently been used to build Spaun, 
  Nengo is the neural simulation software package
  recently used to build Spaun,
  the largest functional brain model to date~\cite{spaun}.
  Despite Nengo being open source, it has not been
  widely used or extended. Two reasons
  for this are the complexity of
  the implementation and the speed of larger simulations.
  In this paper, we describe a new version of Nengo
  which has been reimplemented in Python to
  address these problems.

  Nengo was originally a Java application that
  was extended to use Python scripting through Jython,
  a Java implementation of the Python language~\cite{nengo}.
  While this enabled models to be created
  with simpler syntax, 
  the approach had several weaknesses.
  % there were several weaknesses of the approach. 
  The scripting layer
  could not leverage the Python ecosystem
  because most scientific packages call C or Fortran code.
  Nengo developers had to be familiar with both Java and Python.
  The latest stable version of Jython
  lags behind CPython, so language features like
  decorators could not be used until recently.
  The new implementation of Nengo targets CPython,
  and therefore can take advantage of modern language features
  and libraries like NumPy and Matplotlib.
  The API was also redesigned to provide
  a much simpler interface for modelers.
  Currently, Nengo implements the majority
  of the commonly used features of the Java version,
  with an order of magnitude less code
  (previously $\sim$30,000 lines of Java;
  currently $\sim$2000 lines of Python).

  Nengo now leverages the Theano package~\cite{theano}
  to speed up the creation and simulation of
  Nengo models.
  Theano offers a simple interface for
  describing computations on matrices.
  It builds a graph of those computations,
  simplifies wherever possible, and then transparently
  generates and runs optimized C code.
  Theano can target multi-core CPUs, GPUs,
  and other local and remote computing resources
  with minimal modeler effort.
  Preliminary speed tests have shown that simple models
  can be simulated 9--12~times faster
  using a single CPU core compared to using multiple cores
  with the Java version.
  Simulating a leaky integrate-and-fire neuron in Theano
  can be done $\sim$15~times faster on a GPU than
  on a CPU, and so we predict $\sim$100 times speedups
  for some Nengo models using a GPU.

  %Nengo is now built such that
  %it can use multiple backends to construct and simulate
  %neural models. 
  Nengo now accomodates multiple backends 
  to construct and simulate neural models.
  This will allow, for example,
  a backend that generates PyNN scripts~\cite{pynn}.
  Such a backend would enable other modelers to create
  large-scale functional models by using Nengo
  as a ``neural compiler'' that translates mathematical
  models to biologically realistic spiking neuron models
  without requiring them to learn a new simulation framework.

\begingroup
\renewcommand{\section}[2]{}% Removes the References title
{\scriptsize
\begin{multicols}{2}
\bibliographystyle{unsrt}
\bibliography{biblio.bib}
\end{multicols}}
\endgroup


\end{document}
